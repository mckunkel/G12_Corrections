\documentclass[11pt,a4paper]{article}
\usepackage[T1]{fontenc}
\usepackage[utf8]{inputenc}
\usepackage{authblk}
\usepackage[english]{babel}
\usepackage{fancyhdr}
\usepackage{subfig}
\usepackage{floatrow}
\usepackage{float}
\usepackage{amsmath}
\usepackage{amssymb}
\usepackage{slashed}
\usepackage{graphicx}
\usepackage{todonotes}
\usepackage[toc,page]{appendix}
\usepackage{hyperref}
\usepackage{placeins}
\usepackage{cleveref}
\usepackage{multirow}
\usepackage{longtable}
\usepackage[final]{pdfpages}
\usepackage[export]{adjustbox}
\usepackage[final]{pdfpages}
\hypersetup{
	colorlinks,
	citecolor=purple,
	filecolor=black,
	linkcolor=blue,
	urlcolor=black
}
\usepackage{color}

\newcommand{\white}[1]{{\textcolor{white}{#1}}}



\newcommand*\samethanks[1][\value{footnote}]{\footnotemark[#1]}
%\title{Transition Form Factor of the $\eta^{\prime}$ Meson with CLAS12}
\title{Measurement of Cross-Sections of exclusive $\pi^{0}$ Photo-production on Hydrogen from 1.1 GeV - 5.45 GeV using $\lowercase{e}^{+}\lowercase{e}^{-}\gamma$ decay from the CLAS/g12 Data}
\date{}

\author{Michael C. Kunkel\thanks{email: m.kunkel@fz-juelich.de}\thanks{Now at Forschungszentrum J\"ulich, J\"ulich (Germany)} \\ \vspace{0.3cm} \it \qquad Old Dominion University, VA (U.S.A.) \newline \newline}


\renewcommand\Authands{, }
\fancypagestyle{firststyle}
{
	\fancyhf{}
	\renewcommand{\headrulewidth}{0pt}
	\fancyhead[R]{\small CLAS-NOTE 2017-005}
}
\newlength{\figwidth}
\setlength{\figwidth}{0.9\columnwidth}

\newlength{\qfigheight}
\setlength{\qfigheight}{0.25\textheight}

\newlength{\hfigheight}
\setlength{\hfigheight}{0.5\textheight}

\def\piz{\pi^{0}}
\def\pizT{$\pi^{0} \ $}
\def\pizDal{$\pi^{0} \rightarrow e^+e^- \gamma  $}

\def\etaT{$\eta $}
\def\etaDal{$\eta \rightarrow e^+e^- \gamma  $}

\def\omT{$\omega  $}
\def\omDal{$\omega \rightarrow e^+e^- \piz $}

\def\etaP{\eta^{\prime}}
\def\etaTP{$\eta^{\prime}  $}
\def\etaPDal{$\eta^{\prime} \rightarrow e^+e^- \gamma  $}

\def\phiT{$\phi  $}
\def\phiDal{$\phi \rightarrow e^+e^- \eta  $}
\def\phiDalT{\phi \rightarrow e^+e^- \eta  }

\def\epemT{$ e^+e^-  $}
\def\epem{e^+e^-}
\def\pipiT{$\pi^+\pi^-$}
\def\pipi{\pi^+\pi^-}


\def\phiPR{$ep\to e'p \phi \rightarrow p e^+e^- \eta$}
\def\etaPR{$ep\to e'p \etaP \rightarrow p e^+e^- \gamma$}

%\def\grpath{figures}
\def\figures{/Users/michaelkunkel/WORK/GIT_HUB/THESIS/figures/print}
\newcommand{\abbr}[1]{\textsc{\texttt{#1}}}
\newcommand{\abbrlc}[2]{\textsc{\texttt{#1}}\texttt{#2}}
%\input{variables}
% Document starts
\begin{document}
	\setcounter{page}{3}
%	\includepdf[pages=-]{CheckList_II_final.pdf}
	\maketitle
	\thispagestyle{firststyle}
\begin{abstract}
	Photoproduction of the $\pi^0$ meson was studied using the \textsc{\texttt{CLAS}} detector at Thomas Jefferson National Accelerator Facility using tagged incident beam energies spanning the range $E_{\gamma}=$~1.1~GeV~-~5.45~GeV. The measurement is performed on a liquid hydrogen target in the reaction $\gamma p\to pe^+e^-(\gamma)$. The final state of the reaction is the sum of two subprocesses for $\pi^0$ decay, the Dalitz decay mode of $\pi^0\to e^+e^-\gamma$ and conversion mode where one photon from $\pi^0\to \gamma\gamma$ decay is converted into a $e^+e^-$ pair. This specific final state reaction avoided limitations caused by single prompt track triggering and allowed a kinematic range extension to the world data on $\pi^0$ photoproduction to a domain never systematically measured before.
	
	We report the measurement of the $\pi^0$ differential cross-sections $\frac{d\sigma}{d\Omega}$ and $\frac{d\sigma}{dt}$. The angular distributions agree well with the SAID parametrization for incident beam energies below 3~GeV, while an interpretation of the data for incident beam energies greater than 3~GeV is currently being developed.
\end{abstract}
	\newpage
	\tableofcontents
	\newpage
	\input{intro}
	\input{event_section}
	\input{ANALYSIS/timing}
	\input{ANALYSIS/target_density}
	\input{ANALYSIS/flux}
	\input{ANALYSIS/track_efficiency}
    \input{simulation}

	%\input{Measurement}
	%\input{Manpower}
	\input{results}

	
	\newpage
	\clearpage
	\phantomsection 
	\addcontentsline{toc}{section}{BIBLIOGRAPHY}
	\bibliographystyle{unsrt}
	\bibliography{PI0}	
	
%	\newpage
%	\addcontentsline{toc}{section}{APPENDICES}
%	\let\oldaddtocontents\addtocontents \renewcommand{\addtocontents}[2]{}
%	\begin{appendices} 
%		%\input{Appendix}
%	\end{appendices}
\section{Appendix I}
\subsection{Acceptance plots} \label{sec:Appen}
\includepdf[pages=-]{/Users/michaelkunkel/WORK/GIT_HUB/Pi0_Papers/ANALYSIS_NOTE/REVIEW/AcceptanceAll.pdf}
\subsection{Acceptance plots} \label{sec:AppenII}
\begin{figure}[h!]\begin{center}
		\includegraphics[width=1.1\figwidth,height=1.5\hfigheight]{/Users/michaelkunkel/WORK/CLAS/CLAS6/CODES/SVN/clas/PI0/EGammaVsMomemtumPlots_new.pdf}
		\caption[Mometum depdence on Egamma]{\label{fig:syserrLorentzI}Momentum vs. E$_\gamma$ for proton (top), $e^+$ (middle), and $e^-$ (bottom).}
\end{center}\end{figure}
\begin{figure}[h!]\begin{center}
		\includegraphics[width=1.1\figwidth,height=1.5\hfigheight]{/Users/michaelkunkel/WORK/CLAS/CLAS6/CODES/SVN/clas/PI0/EGammaVsThetaPlots.pdf}
		\caption[Mometum depdence on Egamma]{\label{fig:syserrLorentzII}$\theta_{lab}$ vs. E$_\gamma$ for proton (top), $e^+$ (middle), and $e^-$ (bottom).}
\end{center}\end{figure}
\begin{figure}[h!]\begin{center}
		\includegraphics[width=1.1\figwidth,height=1.5\hfigheight]{/Users/michaelkunkel/WORK/CLAS/CLAS6/CODES/SVN/clas/PI0/ThetaVsMomemtumPlots.pdf}
		\caption[Mometum depdence on Egamma]{\label{fig:syserrLorentzIII}$\theta_{lab}$ vs. Momentum for proton (top), $e^+$ (middle), and $e^-$ (bottom).}
\end{center}\end{figure}
\end{document}